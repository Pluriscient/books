\documentclass[12pt,a4paper]{article} % Could set draft on here for more speed but no images
\usepackage[utf8]{inputenc}
\usepackage[english]{babel}
\usepackage{amsmath}
\usepackage{enumitem}
\usepackage{todo}
\usepackage{hyperref}
\hypersetup{
    colorlinks=true,
    linkcolor=blue,
    filecolor=magenta,      
    urlcolor=cyan,
}
\usepackage{amsfonts}
\usepackage{amssymb}
\usepackage{graphicx}
\usepackage[left=2cm,right=2cm,top=2cm,bottom=2cm]{geometry}
\author{Eric van der Toorn}
\title{Summaries}
\begin{document}
\maketitle
\section{Introduction}
I'll start with an introduction of my general ambition with this document, which is quite simply to record a personal summary and reflection for each of the books I've read or listened to. I have some catching up to do, and I'll start with outlining A. Why I'm doing this, and B. Which books I still have to do this for. Then I'll start with at least 2 of the books within this same paper before creating a structure, because I know from myself that I can get quite distracted/ dwell away from the main point if I do that first.
\subsection{Reasons}
This has been a thought in the back of my mind for a while, and it actually came up quite recently when I was having a discussion with Timotheos, a guest lecturer for the Growth: Managing Your Firm course. We were talking about some of the books we had read and I mentioned how it could get quite mixed up. Then he started talking about how he had a document which collected the books he had read (or at least a subset of them) with summaries and reflections. I agreed that it would be quite a useful thing, and he expounded with another reason: even if you stop with it, a few years later you may reread one of the books and close it with a different viewpoint, which you can then further reflect on. He also mentioned that the document he uses is not intended for modification, only being open for extension.
In general, I can tell for myself that the books I've read that I find useful get clearer in my memory when I focus on my specific memories of reading them, and that I actually feel excited for doing this kind of summary, so hopefully we'll have some fun doing this. I find that the more I write the more I am excited to write more, so let's continue!

\section{List of books}
It is hard to decide which books to do first, but I actually think I will treat it as a stack $->$ Last In First Out, as that way I can discuss the things that are most recent to me and best in my memory first. 
First a list of books that I am currently reading or want to read because they keep popping up in my head:
\subsection{Reading list}
\begin{itemize}
\item 12 Rules for Life (Jordan B. Peterson) - Quite the interesting guy talking about rules you can and should apply to your life
\item Predictable Irrational (Dr. Dan Aley) - Outlining some very interesting things in which humans are ever so irrational.

\item The Great Ideas of Psychology (Daniel N. Robinsson) [ON HOLD] - A summary of psychology from its very roots to its current state.
\end{itemize}
\subsection{Wish list}
\begin{itemize}

\item Black Swan, the impact of the highly improbable (...) [UNSTARTED] - ...
\item Zero to One - Dad really recommends this one
\item 50 lessons I learned from the world (Vic Johnson) - Interesting title and dad finished it
\item De meeste mensen deugen - Dad basically really admires the book
\item Homo Deus - All the other books of Harari have been great
\item Awaken the Giant within | Giant Steps (Tony Robbins) - Want to read and do this one during a vactation or when I have a span of free time large enough.
\item Nudge - Seems like a good book, was quite recommended
\item Find out anything from anyone ...

\end{itemize}
\subsubsection{Considering}
A list of books I'm considering, starting with the books recommended by Nadav (from Growth - Managing Your Firm)
\begin{itemize}
\item At the mind’s limits (Jean Amery)
\item Man’s search for meaning (Viktor Frankl)
\item Fear of freedom (Erich Fromm)
\item In the realm of hungry ghosts (Gabor Mate)
\item The abolition of man (CS Lewis)
\item I and Thou (Martin Buber)
\item Sovereignty of goodness (Iris Murdoch)
\item The malaise of modernity (Charles Taylor)
\item Intimate history of humanity (Theodor Zeldin)
\end{itemize}

\subsection{Read books}

\begin{itemize}
\item Mindset: The new Psychology of success. By Carol S. Dweck, a book expounding the Fixed and Growth mindsets.

\item Crucial Conversations, tools for talking when the stakes are high (Kerry Patterson, Joseph Grenny, Ron McMillan, and Al Switzler), a review on how the authors discovered that one of the major factors that makes people succeed in getting things done, both in business and elsewhere, was the ability to have crucial conversations.
\item Cognitive Behavioral Therapy (Dr. Sattersfield) -  A course on CBT, but still good enough that I feel like it requires a summary in order to remind myself of its major points
\item The 7 Habits of HIghly Effective People (Stephen R. Covey) - A book in which the author names and explains seven habits that are useful to apply, resulting in becoming more effective as a person. Interesting side note: first three are about the self, last four about interacting with others. You have to mold yourself before increasing your effectiveness(?) with others.
\item How to Talk to Anyone (Leil Lowndes) - 92 tips and tricks to interacting with others, some of which can actually be quite useful and fun to apply ;) Side note: will probably list all the tips shortly
\item The Wellbeing Quotient - Elements that contribute to my wellbeing
\item Leadership in a Complex World - A course I followed at the TU Delft as part of the honours programme; really fun and engaging, so I want to document what I learned here.
\item What the most successful people fo before breakfast (Laura Vanderkam) - Taught me the importance of a good morning ritual
\item No excuses  (Brian Tracy) - Quite a simple book on how you shouldn't blame things, just take responsibility or let it go (?)
\item The go-giver - how giving can be way more helpful than just taking - When done the right way
\item 21 lessons of the 21st Century (Harari) - About the general state of the present and the problems we're facing.
\item Homo Sapiens (Harari) - The past of humanity, beautifully given a new perspective
\item The Longevity Code (Kris Verburgh) - Basically my first non-fiction good book that I read for my pleasure and changed my life
\item Atomic Habits (James Clear) - This is slowly but surely changing my life for the better, learning how to approach changing habits.
\item The Subtle Art of not giving a fuck (Mark Manson) - Choose what you care about, don't let your environment choose for you.
\item Becoming (Michelle Obama) - First autobiography I've read afaik, and it was a great experience, think I'm a fan of hers now.
\end{itemize}

\subsection{Books that are complete and summarized}
Awfully empty to start with
\begin{itemize}
\item \ref{mindsets} Mindsets (Carol S. Dweck)
\item Talking To Strangers, (Malcolm Gladwell), a discussion on why we are so bad at perceiving people we don't know much about.
\item Thinking, Fast and Slow (Daniel Kahlemann) - Separates the mind into two parts, a fast intuitive mind and a slow rational one, and shows some very interesting aspects on the difference.
\end{itemize}


\section{Summaries}

\input{summaries/mindsets-dweck.tex}

\subsection{Talking to Strangers (Malcolm Gladwell)}

I did not expect the content I got when I read this book. I expected it to be about some common fallacies we possess and tips on how to deal with them, but instead got a rollercoaster ride emotionally about how 

\subsection{Thinking, Fast and Slow (Daniel Kahnemann)}

This book tells us the story of three pairs of (opposing) entities: That of System 1 and System 2, Econs and humans, and the experiencing and remembering selves.
System 1 and 2 are personalisations of two kinds of ways our brain approaches the world, through an intuitive understanding of the world and speedy decisions (system 1), and a slower, (more) rational system 2, which monitors system 1 and may take over when it doesn't 'agree' or is triggered. Econs are completely rational beings who should be left free to make their own choices, Humans are not and should be given help when making some decisions (tldr; read nudge for this part)
The experiencing self is the human that experiences each moment while the remembering self remembers that and is usually more powerful in making decisions
It is a book about behavioural economics though I found it to be more fitting to decision psychology, as it pertains to how we make decisions and think about them.
The main theme of the book is how we're not rational (though Kahnemann wants to make sure we don't interpret that as meaning we're irrational).
Personally I found many of the examples in the book to be quite insightful and they stimulated me towards reflecting on my own perspectiv on the matters, so I'll take a few to highlight what I mean;


\begin{itemize}[leftmargin=4cm]
\item[Heuristics] When we get a question that is harder to answer and would require quite some thinking our intuitive system (system 1) will usually answer a simpler question and propose its solution instead. Think for instance if you are asked whether you are happy with your life it should take quite a while to collect all the factors which can influence that, but we can also just fill in whether we are happy right now.
\item[The law of small numbers] Extreme cases are more likely to occur when you're looking at small numbers
\item[Pain] Would it be more painful an experience to add onto a torture with the same peak pain an additional 10 minutes of a bit lesser pain? No, it would actually reduce the amount of pain someone remembers (remembering selves have no concept of time)
\item[Asian Disease problem] The way we frame two treatments will have influence on which solution will be chosen which has a paradoxical effect.
\item[Storylines] If you add 10 slightly worse years to someone's life they'll think the entire life's worse than without those years, because the end matters so much.

\end{itemize}


\subsubsection{Concepts named}
As this book is meant to be about learning the concepts and being able to use them, I would be remiss if I did not list them all out to increase the ability of myself to recall them :)

\begin{itemize}[leftmargin=4cm]
\item[Anchoring Effect] When we compare things any numbers that are given are used as a relative setpoint. e.g. If we see a house on sale for 2 millions we'll go lower from there not get an entire different start bid like 2 tons. 
\item[Sunk Cost] Basically the gambler's fallacy, we shouldn't consider what we have already spent when making a choice, just the current state of things.
\item[Loss Aversion] All things considered, losses weigh more than gains both in the amount of emotion they cause and the amount of regret they spark. As such we tend to avoid sure losses and take gambles, while the opposite is the case for gains.
\item[Regression to mean] Luck/chance has a larger role in life than we'd usually like to imagine, which ends up to mean that if something goes well occasionally, that it will also go bad occasionally.
\item[Framing bias] Framing a question one way or another can influence how it is perceived (as a positive or negative outcome, good or bad solution)
\item[Availability bias] When asked how frequently something occurs, you'll think it occurs more frequently if it comes to mind more easily (think terrorist attacks)
\item[Cognitive load] We tend towards less cognitive load (stress on the brain). If something takes a lot of brain power, we'll want to avoid it more.
\item[The Confirmation bias] We tend to look for info that reinforces our own beliefs instead of adapting your beliefs to other facts which we tend to overlook.
\item[Correlation != Causation] No more need be said here
\item[Substitution] Answer a harder question with an easier one (sometimes without knowing)
\item[Causes $>>$ Facts] Humans don't want statistics, they want a sappy story.
\item[Halo effect] Kinda seeing things as black and white. Your first impression colours the story you draw of someone.
\item[Environmental bias] Your surroundings effect you more than you'd want to admit. A lot of posters of the great leader will tend to make you more loyal, sad to say.
\item[Less is more] If you add more detail to a situation you make it seem more likely while you actually reduce the probability (subsets!)

\end{itemize}

\paragraph{More info?}
Could always re-read the book, and I found that the following medium article was quite clear: \href{https://medium.com/swlh/every-chapter-of-thinking-fast-and-slow-in-7-minutes-5e6adf89cf39}{Every Chapter of Thinking Fast, and Slow in 7 Minutes}
\begin{figure}
\includegraphics[scale=0.5]{summaries/thinking-fast-and-slow-systems.png}
\caption{The systems of the mind (useful overview)}
\end{figure}

\end{document}