\subsection{Talking to Strangers (Malcolm Gladwell)}

I did not expect the content I got when I read this book. I expected it to be about some common fallacies we possess and tips on how to deal with them, but instead got a rollercoaster ride emotionally about how some fundamentally human flaws about understanding others contribute to a lot of despairingly solvable problems in our society. 
If I were to give the most powerful one to realise that would be the power of coupling, how certain actions are linked quite tightly with their environment and context. The example that Gladwell mentions is one that will stay with me for a very long time: The suicide of middle-aged women in England dropped significantly after one of the facourite methods for it was replaced (the town-gas that had carbon-monoxide inside and would lead to a 'clean' death by asphyxiation was replaced by modern gas which does not). Or how 90\% of the crime in a city usually centers around 5\% of the streets. Very eery to realise how we can make real life changing decisions like suicide or crime, because of a particular context. 

Gladwell wants us to realise that we are paradoxical in how we handle Strangers; for we understand/know that we ourselves are hard to read and understand but expect to exactly that for others.  To show this he took one of a lot of the psychological analyses that are quite vague and interpretable, like those 'What image do you see in this inksplash', and made some people do a 'complete the word' experiment where you would get a partial word and have to complete it, e.g. \_ \_ \_ v e $ \rightarrow $ k n i v e. When you ask people if you can draw conclusions about *themselves* they will well you that is no way to say anything about them, but if you ask them to do the exact same thing with someone else's results, they will happily draw the most extreme of conclusions. (this also ties into 'what you see is all there is' from Thinking, Fast and Slow)

When people perform counterintuitively to what our culture expects, they are quite quickly ousted or seen as 'strange', like the american student that got convicted (falsely) for murder because she had a different way of coping then she was supposed to have. 

One more interesting aspect that the book also touched upon is one that I believe is quite known to many in some ways: Alcohol transforms people, it doesn't reveal people. It was quite impressive to see that it actually made you adhere *more* to your environmental expectations instead of rebelling which is more in line with what I would have intuitively expected.

In conclusion, this book had a lot to tell, and tell it did. While some of the examples were more gruesome and criminal than others (we essentially went through a list of criminal cases), the way Gladwell extracts a message from these cases is quite compelling, though you should keep in mind that this book is mostly about awareness of our problems, not a set of solutions that we can always use. As always, the first step towards improving at something is to become aware of your current state and where the road forwards is. And in that I believe that Gladwell did quite well.