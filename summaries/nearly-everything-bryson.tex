\subsection{A short History of Nearly Everything, Bill Bryson}

This is Bryson's attempt to explore the natural sciences from their roots to their current state, mostly through the people that have advanced it throughout the past few centuries.
I was expecting a different kind of book when I first received it, with a cover indicative of engineering and design.
Expectant of a descriptive explanation of how everything in the world works, I instead opened it to find an entirely different perspective on the sciences.
While diving through astronomy, physics, tectonics, and biology, Bryson manages to author a compelling scene of how surprising the world of science has been over the years.

I can, with confidence, say that my admiration of many well-known scientists like Newton and Einstein has dimmed and been made more realistic, while my admiration for the adaptability and robustness of the scientific community as a whole has increased, however much they may prove stubborn in many perspectives, the consensus of science has formed, been shaken, broken, and been turned upside down, and then proceeds to be formed once again, now from a different perspective.
It reminds me of the adage that science advances `one funeral at a time' (I believe that was a quantum physicist who  first said it), which while rude and discounting of adaptability still implies that science does not regress, and throughout this book I have not seen a solid countexample to this (though repression of solid ideas until the favoured theorem is toppled is an ongoing theme).
What also impresses me is the amount of time and effort that scientists can dedicate to studying their specialization, as put into perspective by Bryson.
While I am a scientist myself (currently doing a MSc.), I simply cannot muster the dedication required to one-sidedly study something like worms for the entirety of one's career.
Overall the storytelling style of the book is keenly aware of the extraordinariness of whatever the author comes across which highlights these happenings once again.

Overall, an amazing book that can really tell a story of how most of the natural sciences came to be.

On a different note, this book focuses entirely on the western world and its influence, completely putting aside the rest of the world, which is quite unfortunate and calls for a book that also takes into account those contributions.
I rationalize this negligence with Bryson's focus on recent events within a specific subset of the sciences that he found most interesting, though I find it regrettable.
