\subsection{De meeste mensen deugen (Rutger Bregman)}

I'm going to be honest here, I never expected to be putting a Dutch book on this list. Not because I can't read Dutch 
(I have a good grasp of the intricacies of the Dutch language as a native), but because I had grown to hate Dutch literature in all of its forms over the years in high school where I was forced to read and analyze literature that I intensely disliked, which took away my interest. "Is this the best that Dutch literature can offer me," is the thought that recurred time and again. 
My prime example, by virtue of its still-living memory, is not a particularly horrific one but a work that started off great. 'Het Leven in een dag' is a novel that describes how life could look like if we were to live in a world that where lifespans where compressed to a single day, and everything was about 'firsts,' from the first time you saw the sun rise to the first time you had sex. Firsts were also lasts however, and most things could never happen again (we conveniently exclude things like breathing and thinking). In this world, heaven is a single moment where 'everything' happens, and hell is the endless repetition (or 'Earth' for those less religiously inclined) The concept is quite interesting and sets one to thinking about how the repetition of events makes them lose a lot of their magic. The plot of the story was that of a typical romance with an interesting plan: Have more sex by going to hell. In our world the reverse is usually preached, but I like this version more to be completely honest. The plan to go to hell by murdering someone and getting executed succeeds and our main character finds himself on Earth and looks around for his girl.

Then, to completely gore the entire thought process that the first two-thirds of the book stimulate one too, the main character is recruited by the man he killed to become, of all things, a prostitute. The following few scenes are just raunchy depictions of the things he is all but forced to participate in, my 'favourite' being where he has an orgy while skydiving, laid out in explicit detail. 

I realise that I have in the process of wanting to rant about it written an entire review, so I'll now start talking about the book that (after a lot of stimulation from my dad and it being suggested by a lady I met while traveling) I reluctantly started listening to and now am determined to finish to the end, though I sometimes have to bite through some reluctance. 

The book starts off discussing how humanity is perceived, from a philosophical perspective. The age-old question that Bregman wishes to touch upon once again (and hopefully answer) is whether civilization - rules, order, and leadership - is a boon to humanity, a requirement for proper treatment of others, or a poison that has only eroded the basic human decency that is inherent to all of us.

Bregman wants to approach this topic, which is prone to being quite subjective, as scientifically as possible. I believe he did a remarkable job, though the book \textbf{is} named after his opinion, which is that humans are inherently decent. He starts by analyzing our behaviour without civilization, through pre-historical findings, current tribes, and stories where civilization got a 'fresh start' - a la The Lord of the Flies - then continues onto social psychological conclusions that are widely known but deeply controversial, to finish with how things should change if the worldview that most people are decent would be implemented, and what would be the effect. In the conclusion he briefly describes 10 lessons he's learned over his research.

In his efforts to sound rational and scientific he has done quite well, though some parts of the book leave me wondering whether he sufficiently defuses the arguments against human cruelty being non-inherent. One thing that I feel he skirts about unwilling to actually dive into is the belief that civilization is responsible for all that is cruel. He repeatedly paints the civilized world as the 'evil' guy, with its colonization and slavery practices, through the comparisons feel more like comparing apples to oranges. I would've enjoyed it if he'd mentioned more cases where one civilization compared to another. Are leaders always the people with the most hate for eachother? How does this work in rebellions? How should we look at people that commit not violent crime but horrible deeds like paedophilia? These are just some of the questions that the book neglected to touch upon in what felt like an effort to not get into the intricate debates surrounding those topics. 

To conclude my review, this book has made a significant impact on how I perceive the world in terms of how people are, baring some of the resolutions that I felt are intuitive, and with that allowing me to become more resolved in their veracity. On the other hand, it is like an opening move at a debate, where the response and the ensuing persuasive efforts still have to follow. And I am looking forward to see the arguments from both sides.