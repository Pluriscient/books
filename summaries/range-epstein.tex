\subsection{Range: Why Generalists thrive in a Specialized World (David Epstein)}

This was one of those nice reads that makes everything it says stand out as well-founded.
The main point that the author tries to make is that you should not specialize (early),
as it can be just overvuew that you gain by keeping a broader perspective that allows you to succeed.
He illustrates this with several topics: Sports, Art, Music, Study (choice), and (academic) research.
The 'antagonist' is presented in the introduction: the 'kind' world of Tiger Woods, who succeeded because he specialized early.
Epstein argues that this is the exception rather than the rule, and there are many more stories of athletes with 'range': 
They initially explore a braoad range of sports and only after some time (quite late by some measures), fo they delve into one and invest.
Some other quick main points:
\begin{itemize}
\item Start broad, specialize late $\rightarrow$ T-shaped learning
\item Use analogical thinking (like Kepler)
\item Communication is key (an example of the shuttle crash from NASA is given), and don't become too quantative.
\item Plan for short-term not long term (example of a famous female CEO is given, who got where she is by constantly planning short-term)
\begin{itemize}
    \item I'm not too sure how to think of this, as it kinda also invalidates everything about following your dreams.
\end{itemize}
\item Do what you feel passion for now, no when to change. Doesn't mean giving up on having a bad day.
\item Creativity can be found in applying old things to new problems, or new things to old problems,
\item Learn to estimate (like someone from the Manhattan project)
\item We live in a wicked world, not a kind one (reference to Kahnemann)
\item We're getting better at abstract thinking, learn more about Reasoning and Logic.
\end{itemize}

One thought that I got left with is what this entire story implies for the role of guidance. 
If you must let everyone explore as much as they want, can you still guide your e.g. children in some way? 